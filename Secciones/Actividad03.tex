\section{Actividad No 03 – Sistemas de Control de Versiones Libres} 
		
\begin{enumerate}[1.]
	\item CVS
	\\
	\\CVS ha estado durante mucho tiempo, y muchos desarrolladores están ya familiarizados con él. En su día fue revolucionario: fue el primer sistema de control de versiones de código abierto con acceso a redes de área amplia para desarrolladores, y el primero que ofreció ""checkouts"" anónimos de sólo lectura, los que dieron a los desarrolladores una manera fácil de implicarse en los proyectos. CVS sólo versiona ficheros, no directorios; ofrece ramificaciones, etiquetado, y un buen rendimiento en la parte del cliente, pero no maneja muy bien ficheros grandes ni ficheros binarios. Tampoco soporta cambios atómicos.
	

	\item SVK
	\\
	\\Aunque se ha construido sobre Subversion, probablemente SVK[9] se parece más a algunos de los anteriores sistemas descentralizados que a Subversión. SVK soporta desarrollo distribuido, cambios locales, mezcla sofisticada de cambios, y la habilidad de ""reflejar/clonar"" árboles desde sistemas de control de versiones que no son SVK. Vea su sitio web para más detalles.
	

	\item Mercurial
	\\
	\\Mercuria] es un sistemas de control de versiones distribuido que ofrece, entre otras cosas, "una completa ""indexación cruzada"" de ficheros y conjutos de cambios; unosprocotolos de sincronización SSH y HTTP eficientes respecto al uso de CPU y ancho de banda; una fusión arbitraria entre ramas de desarrolladores; una interfaz web autónoma integrada; [portabilidad a] UNIX, MacOS X, y Windows" y más (la anterior lista de características ha sido parafraseada del sitio web de Mercurial).
	

	\item GIT
	\\
	\\GIT es un proyecto empezado por Linus Torvalds para manejar el arbol fuente del ""kernel"" de Linux. Al principio GIT se enfocó bastante en las necesidades del desarrollo del ""kernel"", pero se ha expandido más allá que eso y ahora es usado por otros proyectos aparte del ""kernel"" de Linux. Su página web dice que está "... diseñado para manejar proyectos muy grandes eficaz y velozmente; se usa sobre todo en varios proyectos de código abierto, entre los cuales el más notable es el ""kernel"" de Linux. GIT cae en la categoría de herramientas de administración de código abierto distribuído, similar al, por ejemplo, GNU Arch o Monotone (o bitKeeper en el mundo comercial). Cada directorio de trabajo de GIT es un repositorio completo con plenas capacidades de gestión de revisiones, sin depender del acceso a la red o de un servidor central."

	\item Bazaar
	\\
	\\Bazaar está todavía en desarrollo. Será una implementación del protocolo GNU Arch, mantendrá compatibilidad con el procotolo GNU Arch a medida que evolucione, y trabajará con el proceso de la comunidad GNU Arch para cualquier cambio de protocolo que fuera requerido a favor del agrado del usuario.

\end{enumerate}



